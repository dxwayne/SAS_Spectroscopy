\documentclass[letter,11pt,oneside]{article}

%%% (occur "\\(\\\\[a-z]*section\\|appendix\\|input\\|\\<include\\>\\)")

%%\documentclass[11pt,twocolumn]{article}
%%\usepackage[inline]{asymptote}   %% Inline asymptote diagrams
%%\usepackage{wglatex}             %% Use this one and kill others.
\usepackage{color}               %% colored letters {\color{red}{{text}}
\usepackage{fancyhdr}            %% headers/footers
%%\usepackage{fancyvrb}            %% headers/footers
\usepackage{datetime}            %% pick up tex date time 
\usepackage{lastpage}            %% support page of ...lastpage
\usepackage{times}               %% native times roman fonts
\usepackage{textcomp}            %% trademark
\usepackage{amssymb,amsmath}     %% greek alphabet
\usepackage{parskip}             %% blank lines between paragraphs, no indent
\usepackage{shortvrb}            %% short verb use for tables
\usepackage{lscape}              %% landscape for tables.
\usepackage{longtable}           %% permit tables to span pages wg-longtable
\usepackage{url}                 %% Make URLs uniform and links in PDFs
\usepackage{enumerate}           %% Allow letters/decorations for enumerations
\usepackage{endnotes}            %% Enhance footnotes/endnotes
\usepackage{listings}            %% Make URLs uniform and links in PDFs
\pdfadjustspacing=1                %% force LaTeX-like character spacing
\usepackage{geometry}            %% allow margins to be relaxed
%%\usepackage{wrapfig}             %% permit wrapping figures.
%%\usepackage{subfigure}              %% images side by side.
\geometry{margin=1in}            %% Allow narrower margins etc.
\usepackage[T1]{fontenc}         %% Better Verbatim Font.
\renewcommand*\ttdefault{txtt}   %% 
%%\usepackage{natbib}   %% bibitems

%% include background image (wg-document-page-background) 

\usepackage{graphicx}            %% Include pictures into a document
%% (wg-texdoc-inserttikz)


\def\documentisdraft{NOTDRAFT}

%% (wg-texdoc-isdraft)
%% (wg-texdoc-insert-fancy-headers)

%%\usepackage[bookmarks]{hyperref} %% Make huperlinks within a PDF
%%\usepackage{makeidx}             %% Make an index uncomment following line
%%\makeindex                       %%.. yeah this one, too. index{key} in text
%%



\definecolor{verbcolor}{rgb}{0.6,0,0}
\definecolor{darkgreen}{rgb}{0,0.4,0}
\newcommand\debate[1]{\textcolor{darkgreen}{DEBATE: #1} \marginpar{\textcolor{red}{DEBATE} }}
\newcommand{\ltodo}[2]{\marginpar{\textcolor{red}{ACTION: #1}\endnote{#2}}}
\renewcommand{\thefigure}{\thesection-\arabic{figure}}
\newcommand{\menu}{\ensuremath{\;\rightarrow\;}}




%%%%%%%%%%%%%%%%%%%%%%%%%%%%%%%%%%%%%%%%%%%%%%%%%%%%%%%%%%%%%%%%%%%%%%%%%%%%%


\begin{document}


%% (wg-latex-pretty-title-page)
%% (wg-texdoc-titleblock)

\setcounter{section}{0}
\pagenumbering{arabic}

\ifx\documentisdraft\drafttest
\linenumbers    %%%%%%%%%%%%% DRAFT
\fi

\section{Astronomers Telegram  6 May 2019}
Astronomers Telegram  6 May 2019

\url{http://www.astronomerstelegram.org/}

Analog of the CBAT, the Central Bureau for Astronomical Telegrams,
actual telegrams were sent to people.

CBATs carried a high cost of membership , information was essentially
embargoed to their subscribers.  CBAT started soon after the IAU's
creation in the early 1920s ran until until 2015, when the Commissions
were disbanded. http://www.cbat.eps.harvard.edu/


\url{http://www.astronomerstelegram.org/?read=12660}

ATel reports on a number of astronomical events. 

There are several highly specialized groups that may/may not report,
and may report between themselves faster. Some are driven by
researchers that know each other personally.  I know of instances,
where one person observing at one observatory literally sending an
email to another telescope operator asking for a quick peek at a
target. Hey, get me a quick spectra of this thing. The telegram
follows up hours later.

A casual glance on 6 May 2016 in the MT local AM, with ``Show All''
set, there were 3 telegrams for 5 May -- not a lot. Sometimes they
pour in.

One (so far) from 6 May. 

\subsection{Using ATel}

\vspace{-.15cm}
\begin{enumerate}\addtolength{\itemsep}{-0.5\baselineskip}
   \item   directly from their website. I do this over coffee on mornings when
I have scopes looking for something to do.

   \item   At the telescope, heck something exciting might happen

   \item   email digests; Woody gets his at 9:00 AM, I get mine
at 3:00p
\end{enumerate}

\subsection{Query with programs}

It is possible to query ATel from within a program. The main access
script is written in PERL (ugh) so will require some thought.


\subsection{Account creation}

Create an account, enter your email, a confirm link is sent via that
email to you. You may then visit the email link (center top) and fill
out the query. I get my emails at 15:00 local, time to get later
notices and early enough to plan.

Before getting an account, check the boxes and follow along for a
while to get a feel for what happens.

\subsection{Event history in the ATels}

ATels that are related, have a list of previous (if any) ATels issued
w.r.t. that object. You can click back and read about what has led up
to the current point being made.

There are often mag tables etc reporting observations.

There are links outside of the ATel (usually to the home institution)
for more data - like images of spectra, raw data whatever they
want to offer.

Short summary is one of the best places to find ToO (Target of Opportunity)
requests and fast breaking Nova and Supernova reports. 

\subsection{Types of observing}

\vspace{-.15cm}
\begin{enumerate}\addtolength{\itemsep}{-0.5\baselineskip}
   \item  Photometry
\vspace{-.15cm}
\begin{enumerate}\addtolength{\itemsep}{-0.5\baselineskip}
   \item    Variable star (check every week/year)
   \item    Novae - ignore for years ATel and hit it hard (clear)
   \item    narrow-band (low resolution spectroscopy)
   \item    broad-band
\end{enumerate}
   \item  High Cadence Photometry
   Occultations, star flicker

   \item  Spectroscopy:
\vspace{-.15cm}
\begin{enumerate}\addtolength{\itemsep}{-0.5\baselineskip}
   \item    point source
   \item    longslit                                              
   \item    slit-sliding (eliminate download times)
   \item    slit-scanning
   \item    integral field (lots of fibers make an 'image')
\end{enumerate}

   \item  Output:
\vspace{-.15cm}
\begin{enumerate}\addtolength{\itemsep}{-0.5\baselineskip}
   \item    spectra
   \item    data point
   \item    time-series (novae asteroids)
\end{enumerate}

\end{enumerate}

\subsection{Vocabulary}

Time allocation committee - Write a proposal, get a fixed slot in telescopes
   time -- it may rain!

Queue observing -- to overcome some limitations, TAC says we need
   data.  If something happens raise our weight in the queue. EG Dr Imke
   Pater and infrared KECK observations of IO volcanoes. They go on their
   own schedule.

Automated surveys: PTF, ZWF, Catalina, PanSTARRS, space missions

TESS - Transiting Exoplanet Survey Satellite -- looking for lots of
   ground based observations

LIGO - Laser Interferometer Gravitational-Wave Observatory 

Call for Observations - somebody wants more data \\
Target of Opportunity  (ToO) - somebody wants more data

Coordinated Campaign - usually big project or space/based coordinated

Alerts - AAVSO Alert system

\subsection{Background sources: Surveys}

POSS - Puckett Observatory Supernova Survey \\
 -- amateur based 300+ discoveries

Catalina Sky Survey   \\
   \url{https://catalina.lpl.arizona.edu/}

Polarmar Transient Factory - Caltech/Polamar related \url{https://www.ptf.caltech.edu/} \\
Zwicky Transient Facility -  \url{https://www.ztf.caltech.edu/ }

ASASSN - All Sky Automated Survey for Supernova \\
   \url{http://www.astronomy.ohio-state.edu/~assassin/index.shtml}
   
OGLE    - scopes stare at rich patch of sky for microlensing \\
  \url{http://ogle.astrouw.edu.pl/}
 
ICECUBE - Neutrino  \\
\url{https://icecube.wisc.edu/}

LIGO (gravity wave) all hands on deck for the GW170817 event. \\
\url{https://www.ligo.caltech.edu/}

LCO \\
\url{https://lco.global/}

PanSTARRs \\
\url{https://panstarrs.stsci.edu/}

ATLAS The Asteroid Terrestrial-impact Last Alert System
\url{http://www.ifa.hawaii.edu/info/press-releases/ATLAS/}

Various space missions
DAMPE-  (China) The DArk Matter Particle Explorer \\
SWIFT \\
INTEGRAL - INTErnational Gamma-Ray Astrophysics Laboratory \\
NICER  Neutron Star Interior Composition Explorer (NICE too) \\
Chandra \\
NuSTAR

MAXI Monitor of All-sky X-ray Image (MAXI) ISS (Japan) \\
Fermi 

etc...

There are lots of space missions flying now.


-- OK Now you have a tiger by the tail, how do you plan
a set of observations?





%%\appendix
%%\renewcommand \thesection{\Alph{section}}

%% use a bibitem approach to the references publications etc.
%% (wg-bibitem)

%%\clearpage
%%\addcontentsline{toc}{section}{References}
%%\renewcommand*{\refname}{My Bibliography and References}
%%\bibliographystyle{plain}	% bibliographystyle{apalike} and \usepackage{natbib}
%%\bibliography{MasterBib}	% expects file "MasterBib.bib"


%%\clearpage
%%\addcontentsline{toc}{section}{Index}
%%\printindex %% www.cs.usask.ca/resources/tutorials/latex/notes/toc-index.pdf

%%\begin{thebibliography}{80}
%%\usepackage{natbib}   %% bibitems
%%\end{thebibliography}

% /home/wayne/git/SAS_Spectroscopy/doc/atel.tex

%% (wg-texdoc-endnotes)
\end{document}
