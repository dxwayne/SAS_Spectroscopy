From the https://www.physics.nist.gov/PhysRefData/ASD/lines_form.html
site. Use the advanced options.

\begingroup \fontsize{10pt}{10pt}
\selectfont
%%\begin{Verbatim} [commandchars=\\\{\}]
\begin{verbatim} 
   6 068.8+    1.07e+05      D   19.345197   -  19.549496    3s23p2(3P)4d   4P    3/2   3s23p2(3P)5p   4P°   1/2     u20,LS    
   6 451.0+    1.42e+04      E   19.345197   -  19.537390    3s23p2(3P)4d   4P    3/2   3s23p2(3P)5p   4P°   3/2     u20,LS    
   6 540.9+    1.70e+04      E   19.359944   -  19.549496    3s23p2(3P)4d   4P    1/2   3s23p2(3P)5p   4P°   1/2     u20,LS    
   6 987.1+    3.50e+04      D   19.359944   -  19.537390    3s23p2(3P)4d   4P    1/2   3s23p2(3P)5p   4P°   3/2     u20,LS    

Astrophysical S II transitions Indicator of shocks.
  Ritz         Aki          Ei           Ek        Lower Level                 Upper Level     Type           TP 
  Wavelength   (s-1)       (eV)         (eV)      Conf., Term, J              Conf., Term, J                 Ref.
  Vac (nm)    
  6 451.0+     1.42e+04  19.345197 - 19.537390    3s23p2(3P)4d   4P    3/2   3s23p2(3P)5p      4P°   3/2     u20,LS    
  6 540.9+     1.70e+04  19.359944 - 19.549496    3s23p2(3P)4d   4P    1/2   3s23p2(3P)5p      4P°   1/2     u20,LS    
\end{verbatim}
\endgroup
%% \end{Verbatim}

Start with a star;

First off, consider the physics along the line of site. The stellar
core, photosphere, exosphere, ISM, atmosphere (tellurics, aurora),
optics, spectrograph, sensor, analysis software, brain,
publication. (This includes the social and political layers.)

Essentially, for a star, the continuum is set by the photosphere
superimposed with small scale events like flares, convection, magnetic
fields, limb darkening, rotation rate, star-spots, metallicity,
T$_{eff}$, surface gravity.

So,,, 1 measurement (image), too many parameters, like:

T$_{eff}$        - the excitation source for lines
limb darkening  - line assymetry blue is absorbed more than red
rotation rate   - broader lines
star-spots      - de-emphasizes one rotation vector over other
metallicity     - differential absorption, actual apperence in spectra
   molecules (Late K and Ms)
surface gravity - slight shift in lines 

flares          - non-LTE event lots in M-type stars
convection      - broadening 
magnetic fields - Zeeman effect / star spot / internal dynamos


Secondary: other masses in the system.

Other: exosphere - spun out material; left over disc from accretions;
other terpitude from special stars. Then the ISM, opacity, etc.

Picking up at the atmosphere, the usual considerations apply.

I was not kidding about the brain -- misconceptions or under-understanding
of the physics -- we call it science afterall.

Run-of-the-mill situation. HSR star mid-life.

Interesting situations: pre-main sequence, entering the HR diagram.
AGB stars; other than class V stars; pecular stars; multiplicity:
hierarchial/non-hierarchial. Ejected (high proper motion stars).

Most of the interesting bits are way dim. 

Absolute Magnitude: LTE and NLTE

Absorption (BORINGGGGG except when its not, emission).

Non-run-of-the-mill: clouds, jets, nova, SNe, GW events, and...
my new favorite intermediate-luminosity events.

\section{Instrumentation Limits}

ATM window is 3300A 
onset of tellurics ~5000A and 6500A
Light pollution

QE sensor
Color of optics (instrumentals)


Optics:

pixelscale
seeing
resolution $1.22\lambda/D$

Other things to watch out for:

line-blanketing
Hayashi limit for hydrostatic equilibrium (cool Red Supergiants)



Ca II triplet lines at 8498, 8542, and 8662







  



  



